%*=============================================================================*
%* LICENSE
%*=============================================================================*

% Copyright (c) 2022 Mathis Gauthey
% This source code is licensed under the MIT License found in the
% LICENSE file in the root directory of this source tree.

%*=============================================================================*
%* Début du préambule LaTeX
%*=============================================================================*

%*-----------------------------------------------------------------------------*
%* Classe du document
%*-----------------------------------------------------------------------------*

\documentclass[a4paper,11pt]{report}    % A4, police taille 11, report pour avoir les chapitres

%*-----------------------------------------------------------------------------*
%* Langue française pour le document
%*-----------------------------------------------------------------------------*

\usepackage{latexsym}       % Police latex de base
\usepackage[french]{babel}  % Dictionnaire français (indentation, caractères spéciaux, tirets...)
\usepackage[utf8]{inputenc} % Encodage d'entrée pour les caractères accentués
\usepackage[T1]{fontenc}    % Affichage correct des caractères accentués notamment

%*-----------------------------------------------------------------------------*
%* Géométrie du document
%*-----------------------------------------------------------------------------*

% Gestion des différentes marges du document pour coller avec le séparateur d'en-tête
\usepackage[left=2cm , right=2cm, bottom=2cm, top=2cm, headheight=2cm, headsep=0.5cm,heightrounded=true]{geometry}
% \raggedbottom % This makes the last line of each page be at exactly the same place on the sheet of paper
% \flushbottom  % All pages will not necessarily have exactly the same height, but ‘almost the same height’
\setlength{\parskip}{1em}   % Définition de l'espace entre les paragraphes
\setlength{\parindent}{4em} % Définition de la longueur du "tab" = indentation
\usepackage{fancyhdr}       % Permet en-tête et pied de page
\pagestyle{fancyplain}      % Pour avoir le même style sur les pages fancy et sur celles plains comme la toc
\renewcommand\plainheadrulewidth{.4pt}  % Le trait noir sous les logos sur les pages plain
\fancyhead[L]{\includegraphics[scale=0.1]{Logo_Univ_Lyon_1.png}}    % Logo gauche
\fancyhead[R]{\includegraphics[scale=0.15]{Logo_Polytech_Lyon.png}} % Logo droit
% Redéfinir le style "empty" utilisé par le documentclass "report" pour le titre, résumé et chapitres
\fancypagestyle{empty}
{
    \fancyhf{}
    \fancyhead[L]{\includegraphics[scale=0.1]{Logo_Univ_Lyon_1.png}}    % Logo gauche
    \fancyhead[R]{\includegraphics[scale=0.15]{Logo_Polytech_Lyon.png}} % Logo droit
}

%*-----------------------------------------------------------------------------*
%* Liens dans le document, méta-datas et références bibliographiques
%*-----------------------------------------------------------------------------*

\usepackage{xpatch}     % Permet de patcher certaines fonctionnalités de base comme la toc
\usepackage{float}      % Placement des flottants
\usepackage{hyperref}   % Liens dans le documents
\usepackage{caption}    % Légendes dans les environnements "figure" et "float"
\usepackage[list=true]{subcaption} % Légendes pour les "sous-figures" et "sous-float"
                                   % et affichage des "sous-..." dans la liste des figures
\def\thechapter{\Alph{chapter}}    % Définition des chapitres avec une lettre
\setcounter{tocdepth}{3}           % Profondeur de numérotation de la toc  (chap > sec > subsec > subsubsec)
\setcounter{secnumdepth}{3}        % Profondeur de numérotation des titres (chap > sec > subsec > subsubsec)
\usepackage{chngcntr}              % Permet de changer les numérotations d'objets
\usepackage[titles]{tocloft}       % Gestion très précise des différentes listes

\hypersetup % Attribution des méta-datas pdf pour reconnaissance automatique Zotero entre autres
{
pdftitle={Titre},
pdfsubject={Sujet},
pdfauthor={Auteur},
pdfkeywords={keyword1} {keyword2} {keyword3}
}

%*-----------------------------------------------------------------------------*
%* Graphiques, police et maths
%*-----------------------------------------------------------------------------*

\usepackage[table,xcdraw]{xcolor}       % Package permettant d'utiliser de la couleur
% \usepackage{color}                      % Rajouter de la couleur au texte
\usepackage{bm}                         % Mettre en gras des maths avec la commande \bm
\usepackage{ragged2e}                   % Meilleure gestion de l'alignement des textes entre autres
\usepackage{tcolorbox}                  % Boite colorées pour texte, images ou équations
\usepackage{textcomp}                   % Symboles et polices
\usepackage{gensymb}                    % Symbole pour le degré entre autre
\usepackage{amsmath,amsfonts,amssymb}   % Écrire des maths
\usepackage{cancel}                     % Barrer des maths
\usepackage{mathtools}                  % Gestion des matrices et de maths complexes
\usepackage{morewrites}                 % Résout un problème entre les listes équation et codes
\usepackage[color={[RGB]{255,191,191}}, text=Confidentiel, scale=1]{draftwatermark} % Ajout filigrane

\usepackage{lmodern}
\usepackage[Lenny]{fncychap}

\ChNameUpperCase
\ChNumVar{\fontsize{40}{42}\usefont{OT1}{ptm}{m}{n}\selectfont}
\ChTitleVar{\Huge \bfseries}

\usepackage{pifont}
\usepackage{enumitem}       % Gestion des énumérations

% Gestion des espaces avant et après les listes plus agréable à la vue
% \setlist[itemize]{noitemsep, topsep=5pt, before={\vspace*{-\baselineskip}}}
% Si désactivé (commenté), penser à ajouter un \smallskip, \medskip ou \bigskip après un itemize.

\definecolor{bulletcolor}{RGB}{128,128,128}

\setenumerate[1]{label=\Alph*.}
\setenumerate[2]{label=\arabic*.}
\setenumerate[3]{label=\roman*.}

\setitemize[1]{label=\ding{222}, font=\LARGE \color{bulletcolor}}
\setitemize[2]{label=\textbullet, font=\LARGE \color{bulletcolor}}
\setitemize[3]{label=$\triangleright$, font=\LARGE \color{bulletcolor}}

\usepackage{siunitx}                                    % Pour des unités bien écrites
\newcommand{\nomunit}[1]{%
\renewcommand{\nomentryend}{\hspace*{\fill}#1}}         % Commande pour la nomenclature (unités à droite)
\sisetup{inter-unit-product =\ensuremath{{}\cdot{}}}    % Séparation par un point des unités
\DeclareSIUnit\bar{bar}                                 % Besoin de déclarer les bar car pas pris en charge
\DeclareSIPower\quartic\tothefourth{4}

\usepackage{contour}
\usepackage[normalem]{ulem}

\renewcommand{\ULdepth}{1.8pt}
\contourlength{0.8pt}

\newcommand{\myuline}[1]{%
	\uline{\phantom{#1}}%
	\llap{\contour{white}{#1}}%
}

% myuline on each word to allow linebreaks
\RequirePackage{xparse}
\ExplSyntaxOn
\NewDocumentCommand{\myulineX}{m}
{
	\seq_set_split:Nnn \l_tmpa_seq { ~ } { #1 }
	\seq_map_inline:Nn \l_tmpa_seq { \myuline{##1} ~ } \unskip
}
\ExplSyntaxOff

%*-----------------------------------------------------------------------------*
%* Glossaire
%*-----------------------------------------------------------------------------*

\usepackage[acronym,toc]{glossaries}    % Gestion d'un glossaire et d'une liste d'acronymes,
                                        % et ajout dans la toc de la position de ces derniers
% Pas besoin de points à la fin du document

\newglossaryentry{mot_complexe}{
    name={mot complexe},
    description={Un mot complexe nécessite généralement une explication}}

\newacronym{lfala}{LFALA}{Les Français Aiment Les Acronymes}                        % Récupère les informations du fichier glossary.tex
\makeglossaries                         % Génère le glossaire avec les informations récupérées

%*-----------------------------------------------------------------------------*
%* Nomenclature
%*-----------------------------------------------------------------------------*

\usepackage[intoc]{nomencl}         % Gestion d'une nomenclature avec position dans la toc
\makenomenclature                   % Génère la nomenclature
\usepackage{etoolbox}               % Permet de créer des groupes de nomenclature

% Création de groupes de nomenclature : ATTENTION -> Uniquement des caractères uniques en identifiant
\renewcommand\nomgroup[1]{%
  \item[\bfseries
  \ifstrequal{#1}{A}{Groupe 1}{%
  \ifstrequal{#1}{B}{Groupe 2}{%
  \ifstrequal{#1}{C}{Groupe 3}{%
  }}}%
]} % Attention aux accolades lors de la création de groupes

\AtBeginDocument{   % Nomenclature à générer
\nomenclature[A]{$c_{air}$}{Célérité du son dans l'air à \SI{15}{\celsius} \nomunit{\SI{340.29}{\meter\per\second}}}
\nomenclature[B]{$T_N$}{Période propre de l'élément considéré}
\nomenclature[B]{\(E\)}{{Module d'Young}}
\nomenclature[C]{$\dot{\epsilon}$}{Vitesse de déformation \nomunit{\si{\per\second}}}
}

%*-----------------------------------------------------------------------------*
%* Gestion des figures
%*-----------------------------------------------------------------------------*

\usepackage{graphicx}               % Plus d'arguments pour la fonction \includegraphics
\graphicspath{{Images/}}            % Path des images

\counterwithin{figure}{section}     % Numérotation des figures à partir des sections
\setcounter{lofdepth}{2}            % Afficher jusqu'aux sous-figures dans la liste des figures
\cftsetindents{figure}{0em}{3.5em}  % Réglage de l'espace entre le numéro et le nom de la figure dans la liste
\setlength\cftbeforefigskip{5pt}    % Réglage de l'espacement entre les figures dans la liste
\AtBeginDocument{\renewcommand{\listfigurename}{Liste des figures}} % Renommer la liste des figures
% Ajout de la position de la liste des figures dans la toc
\xpretocmd{\listoffigures}{\addcontentsline{toc}{chapter}{\listfigurename}}{}{}

%*-----------------------------------------------------------------------------*
%* Gestion des tableaux
%*-----------------------------------------------------------------------------*

\usepackage{array,multirow,makecell}                        % Packages utiles pour les tableaux
\setcellgapes{1pt}                                          % Paramètres sympa d'après Xm1Math
\makegapedcells                                             % Paramètres sympa d'après Xm1Math
\newcolumntype{R}[1]{>{\raggedleft\arraybackslash }b{#1}}   % Paramètres sympa d'après Xm1Math
\newcolumntype{L}[1]{>{\raggedright\arraybackslash }b{#1}}  % Paramètres sympa d'après Xm1Math
\newcolumntype{C}[1]{>{\centering\arraybackslash }b{#1}}    % Paramètres sympa d'après Xm1Math

\counterwithin{table}{section}      % Numérotation des tableaux à partir des sections
\setcounter{lotdepth}{2}            % Afficher jusqu'aux sous-tableaux dans la liste des tableaux
\cftsetindents{table}{0em}{3.5em}   % Réglage de l'espace entre le numéro et le nom du tableau dans la liste
\setlength\cftbeforetabskip{5pt}    % Réglage de l'espacement entre les figures dans la liste
% Ajout de la position de la liste des tableaux dans la toc
\xpretocmd{\listoftables}{\addcontentsline{toc}{chapter}{\listtablename}}{}{}

%*-----------------------------------------------------------------------------*
%* Gestion des équations
%*-----------------------------------------------------------------------------*

\newcommand{\listequationsname}{Liste des équations}    % Renommer la liste des équations
\newlistof{myequations}{equ}{\listequationsname}
\newcommand{\myequations}[1]{%
   \addcontentsline{equ}{myequations}{\protect\numberline{\theequation}#1}
}
\counterwithin{equation}{section}           % Numérotation des équations à partir des sections
\cftsetindents{myequations}{0em}{3.5em}     % Réglage de l'espace entre le numéro et le nom de l'équation dans la liste
\setlength\cftbeforemyequationsskip{5pt}    % Réglage de l'espacement entre les équations dans la liste

% Création de la commande \noteworhty pour les équations importantes qui méritent d'être listées
\newcommand{\noteworthy}[2]{
\begin{align} \label{#2} \ensuremath{\boxed{#1}} \end{align}
\myequations{#2}
\begingroup
\centering \small \textit{#2}

\endgroup}

\makeatletter   % Espacement des équations plus important entre les chapitres
\xpretocmd{\@chapter}{\addtocontents{equ}{\protect\addvspace{10\p@}}}{}{}{}%
\makeatother
% Ajout de la position de la liste des équation dans la toc
\xpretocmd{\listofmyequations}{\addcontentsline{toc}{chapter}{\listequationsname}}{}{}

%*-----------------------------------------------------------------------------*
%* Gestion des codes
%*-----------------------------------------------------------------------------*

\usepackage[newfloat]{minted}       % Package permettant d'intégrer du code
% Mode printer friendly
\usemintedstyle{emacs}            	% Sélection du style de syntaxe
\definecolor{bg}{HTML}{FFFFFF}      % Définition de la couleur d'arrière plan blanc
% Mode pdf only
%\usemintedstyle{monokai}            % Sélection du style de syntaxe
%\definecolor{bg}{HTML}{282828}      % Définition de la couleur d'arrière plan monokai
\counterwithin{listing}{section}    % Numérotation des codes à partir des sections

% Configuration de newfloat pour que tout roule
\newenvironment{code}{\captionsetup{type=listing}}{}
\SetupFloatingEnvironment{listing}{%
  name={Listing},
  fileext=lol}
\SetupFloatingEnvironment{listing}{listname=Liste des codes}    % Renommer la liste des codes

% Gérer les codes de plus d'une page de document
\newenvironment{longlisting}{\captionsetup{type=listing}}{} % longlisting pour les codes qui dépassent d'une page
\vbadness=\maxdimen    % Casse les couilles ya pas de fix qui garde une couleur de background et une bonne liste
                       % du-coup je décide de supprimer les warnings.
% Source : https://tex.stackexchange.com/questions/50830/do-i-have-to-care-about-bad-boxes/50850#50850
%        & https://tex.stackexchange.com/questions/138/what-are-underfull-hboxes-and-vboxes-and-how-can-i-get-rid-of-them

\setminted          % Paramètre du code standard
{
breaklines,
bgcolor=bg,
frame=lines,
framesep=2mm,
baselinestretch=1.2,
fontsize=\footnotesize,
linenos=true,
autogobble
}

\setmintedinline    % Paramètres du code inline
{
breaklines,
bgcolor=bg,
frame=lines,
framesep=2mm,
baselinestretch=1.2,
fontsize=\footnotesize,
linenos=true,
autogobble
}

%*-----------------------------------------------------------------------------*
%* Index
%*-----------------------------------------------------------------------------*

\usepackage{imakeidx}   % Package pour créer l'index
\makeindex              % Génération de l'index
% Ajout de la position de l'index dans la toc
\xpretocmd{\printindex}{\addcontentsline{toc}{chapter}{\indexname}}{}{}

%*-----------------------------------------------------------------------------*
%* Bibliographie
%*-----------------------------------------------------------------------------*

% Importer un fichier biblatex, sans dépassement des marges, trié par ordre d'apparition
\usepackage[block=ragged,sorting=none]{biblatex}
\usepackage{csquotes}           % Gestion des caractères " " lors des citations
\addbibresource{biblio.bib}     % Importer le fichier de bibliographie
\nocite{*}                      % Importer les éléments non cités quand même dans la bibliographie

%*-----------------------------------------------------------------------------*
%* Gestion des annexes
%*-----------------------------------------------------------------------------*

\usepackage[toc,page,title,titletoc,header]{appendix}   % Packages indexes importants
\usepackage{pdfpages}                                   % Intégration de pdf dans le document
\renewcommand{\appendixtocname}{Table des annexes}      % Nom de la table des annexes dans la toc
\renewcommand{\appendixpagename}{Annexes}               % Nom du titre de la page des annexes
\usepackage{titletoc}	% Permet de générer une petite table des annexes

%*-----------------------------------------------------------------------------*
%* Utilitaires
%*-----------------------------------------------------------------------------*

\usepackage[all,defaultlines=3]{nowidow}    % Macro pour la gestion des lignes seules en bout de page
\usepackage{blindtext}                      % Génération de texte aléatoire pour les exemples
% A utiliser avec https://ctan.mirror.garr.it/mirrors/ctan/macros/latex/contrib/mwe/mwe.pdf pour les images

%*=============================================================================*
%* Fin du préambule LaTeX
%*=============================================================================*

%*=============================================================================*
%* Début du document LaTeX
%*=============================================================================*

\begin{document}

%*-----------------------------------------------------------------------------*
%* Début de la page de garde
%*-----------------------------------------------------------------------------*

\hypersetup{pageanchor=false}
\begin{titlepage}
    \begin{center}
        \vspace*{0.5cm}

        \Huge
        \textbf{Mémoire professionnel de Xe année}\\
        \Large
        présenté par\\

        \vfill

        \Large
        \textbf{Prénom NOM}\\
        \emph{Filière}\\
        Années YYYY-YYYY

        \vfill

        \Huge
        Énoncé du stage\\

        \vfill

        \begin{figure}[H]
          \centering
          \caption*{\Large Logo de l'entreprise et citation de la source}
          \label{fig:Banniere_Meca}
          \includegraphics[width=0.9\textwidth]{Images/Banniere_Meca.png}
        \end{figure}

        \vfill

        \Large
        \begin{flushleft}
            \begin{tabular}{l}
                Tuteurs de l'entreprise : Prénom NOM\\
                Tuteur de l'école : Prénom NOM
            \end{tabular}
        \end{flushleft}

        \begin{flushright}
          Août 2022
        \end{flushright}

        \Large
        \noindent\fbox{\begin{minipage}[c][0.3\textwidth]{\textwidth-7pt}%
        \begin{center}
        POLYTECH LYON - UNIVERSITE CLAUDE BERNARD LYON 1
        \par\end{center}
        \begin{center}
        Domaine Scientifique de La Doua -- 15, Boulevard Latarjet
        \par\end{center}
        \begin{center}
        69622 VILLEURBANNE CEDEX
        \par\end{center}
        \begin{center}
        Tél. (33) 04.72.43.12.24 - Fax. (33) 04.72.43.12.25
        \par\end{center}
        \begin{center}
        http://polytech.univ-lyon1.fr/
        \par\end{center}%
        \end{minipage}}
    \end{center}
\end{titlepage}

%*-----------------------------------------------------------------------------*
%* Fin de la page de garde
%*-----------------------------------------------------------------------------*

%*-----------------------------------------------------------------------------*
%* Début du résumé
%*-----------------------------------------------------------------------------*

% French

\thispagestyle{empty}
\begin{center}
    \Large
    \textbf{Titre}

    \vspace{0.4cm}
    \large
    Sous titre

    \vspace{0.4cm}
    \textbf{Auteur}

    \vspace{0.9cm}
    \textbf{Résumé}
\end{center}

\blindtext

% English

\newpage
\thispagestyle{empty}
\begin{center}
	\Large
	\textbf{Title}

	\vspace{0.4cm}
	\large
	Subtitle

	\vspace{0.4cm}
	\textbf{Author}

	\vspace{0.9cm}
	\textbf{Abstract}
\end{center}

\blindtext

%*-----------------------------------------------------------------------------*
%* Fin du résumé
%*-----------------------------------------------------------------------------*

%*-----------------------------------------------------------------------------*
%* Début TOC & Co
%*-----------------------------------------------------------------------------*

\newpage
\hypersetup{pageanchor=true}
\setcounter{page}{1}
\tableofcontents

\glsaddall  % Comme le \nocite{*} pour la bibliographie, importe toutes les entrées au moins une fois
\printglossaries

\newpage
\printnomenclature

%*-----------------------------------------------------------------------------*
%* Fin TOC & Co
%*-----------------------------------------------------------------------------*

%*-----------------------------------------------------------------------------*
%* Début du corps du document
%*-----------------------------------------------------------------------------*

\newpage
\chapter{Chapitre avec un titre extrêmement long, tellement long que personne n'utilisera jamais un titre pareil mais sait-on jamais}

\section{Section}

\myulineX{Ceci est un test permettant de déterminer si oui ou non la capacité à souligner sans briser le linebreak est fonctionnelle. En effet, Latex de base ne permet pas de réaliser cette manœuvre, mais grâce aux commandes du préambule, le sous-lignage est adaptatif selon les lettres utilisées et de plus est utilisable sur de longues phrases.}

\subsection{Subsection}

\blindtext

\subsubsection{Subsubsection}

\blindtext

\chapter{Tuto pratique}

\section{Les énumérations}

Différence de distance avec le texte :

\begin{itemize}
	\item First
	\item Second
	\item Third
\end{itemize}

Différence de distance avec le texte quand le code est collé :
\begin{itemize}
	\item First
	\item Second
	\item Third
\end{itemize}\bigskip	% Important pour permettre l'espacement optimal du texte avec les listes.

Les énumérations :

\begin{enumerate}
	\item First
	\item Second
	\item Third
\end{enumerate}

Les énumérations collées :
\begin{enumerate}
	\item First
	\item Second
	\item Third
\end{enumerate}

\newpage

Si maintenant on veut s'amuser avec des subitems :
\begin{itemize}
	\item Item 1
	\subitem Subitem 1
	\subitem Subitem 2
	\item Item 2
	\subitem Subitem 1
	\subsubitem Subsubitem 1
	\subitem Subitem 2
	\item Item 3
\end{itemize}

Et avec les énumérations :
\begin{enumerate}
	\item Item 1
	\subitem Subitem 1
	\subitem Subitem 2
	\item Item 2
	\subitem Subitem 1
	\subsubitem Subsubitem 1
	\subitem Subitem 2
	\item Item 3
\end{enumerate}

Et si on joue avec les listes imbriquées :
\begin{itemize}
	\item First
	\begin{itemize}
		\item Item
		\item Item
		\begin{itemize}
			\item Item
		\end{itemize}
	\end{itemize}
	\item Second
	\item Third
\end{itemize}

Et si on joue avec les énumérations imbriquées :
\begin{enumerate}
	\item First
	\begin{enumerate}
		\item Item
		\item Item
		\begin{enumerate}
			\item Item
		\end{enumerate}
	\end{enumerate}
	\item Second
	\item Third
\end{enumerate}

\newpage

Les descriptions :
\begin{description}
	\item[CMS :] Un système de gestion de contenu ou SGC...
	\item[Framework :] En programmation informatique, un framework....
	\item[SGBD :] Un système de gestion de base de données...
\end{description}

\section{Les acronymes et les glossaires}

C'est fou, on peut imprimer les entrées de glossaire de cette manière pour que sa position soit identifiée dans le glossaire : \gls{mot_complexe}

Et pour les acronymes c'est encore mieux, on a plein d'options:
\begin{itemize}
    \item \acrshort{lfala}
        \item \acrlong{lfala}
    \item \acrfull{lfala}
\end{itemize}

\section{Les figures et sous figures}

\begin{figure}[H]
    \centering
    \includegraphics[width=0.25\textwidth]{example-image-a}
    \caption{Image A}
    \label{fig:example-image-a}
\end{figure}

On peut même les référencer : figure \ref{fig:example-image-a}

\begin{figure}[H]
    \begin{subfigure}[t]{0.475\textwidth}
        \includegraphics[width=1\textwidth]{example-image-b}
        \caption{Image B}
        \label{subfig:example-image-b}
    \end{subfigure}%
    \hfill
    \begin{subfigure}[t]{0.475\textwidth}
        \includegraphics[width=1\textwidth]{example-image-c}
        \caption{Image C}
        \label{subfig:example-image-c}
    \end{subfigure}
    \caption{Example d'utilisation des sous-figures}
    \label{fig:test_subfigure}
\end{figure}

\begin{figure}[H]
    \centering
    \includegraphics[width=0.25\textwidth]{example-image}
    \caption{Image}
    \label{fig:example-image}
\end{figure}

\section{Les tableaux}

\begin{table}[H]
\centering
\begin{tabular}{||c c c c||}
 \hline
 Col1 & Col2 & Col2 & Col3 \\ [0.5ex]
 \hline\hline
 1 & 6 & 87837 & 787 \\
 2 & 7 & 78 & 5415 \\
 3 & 545 & 778 & 7507 \\
 4 & 545 & 18744 & 7560 \\
 5 & 88 & 788 & 6344 \\ [1ex]
 \hline
\end{tabular}
\caption{Table to test captions and labels.}
\label{table:1}
\end{table}

\begin{table}[H]
    \begin{subtable}[h]{0.45\textwidth}
        \centering
        \begin{tabular}{l | l | l}
        Day & Max Temp & Min Temp \\
        \hline \hline
        Mon & 20 & 13\\
        Tue & 22 & 14\\
        Wed & 23 & 12\\
        Thurs & 25 & 13\\
        Fri & 18 & 7\\
        Sat & 15 & 13\\
        Sun & 20 & 13
       \end{tabular}
       \caption{First Week}
       \label{subtab:week1}
    \end{subtable}%
    \hfill
    \begin{subtable}[h]{0.45\textwidth}
        \centering
        \begin{tabular}{l | l | l}
        Day & Max Temp & Min Temp \\
        \hline \hline
        Mon & 17 & 11\\
        Tue & 16 & 10\\
        Wed & 14 & 8\\
        Thurs & 12 & 5\\
        Fri & 15 & 7\\
        Sat & 16 & 12\\
        Sun & 15 & 9
        \end{tabular}
        \caption{Second Week}
        \label{subtab:week2}
     \end{subtable}
     \caption{Max and min temps recorded in the first two weeks of July}
     \label{tab:temps}
\end{table}

\begin{table}[H]
\centering
\begin{tabular}{|ll|r|}
\hline
\multicolumn{2}{|c|}{Item}                    &            \\ \hline
\multicolumn{1}{|l|}{Animal}    & Description & Price (\$) \\ \hline
\multicolumn{1}{|l|}{Gnat}      & per gram    & 13.65      \\ \hline
\multicolumn{1}{|l|}{}          & each        & 0.01       \\ \hline
\multicolumn{1}{|l|}{Gnu}       & stuffed     & 92.50      \\ \hline
\multicolumn{1}{|l|}{Emu}       & stuffed     & 33.33      \\ \hline
\multicolumn{1}{|l|}{Armadillo} & frozen      & 8.99       \\ \hline
\end{tabular}
\caption{Exemple supplémentaire}
\label{tab:exemple_sup}
\end{table}

\section{Les équations}

\begin{equation}
    ax^2+bx+c=\Delta
\end{equation}

\begin{align*}  % Le & permet de choisir où on souhaite aligner l'équation
    35x+6&=4\\
    &=z+2
\end{align*}

\noteworthy{P(\bigcup_{n=1}^n A_n) \leq \sum_{n=1}^n P(A_n)}{Boole's inequality}

\noteworthy{E=mc^2}{E=M6 tmtc}

On peut meme les citer : \ref{Boole's inequality}

Oh et évidemment on peut mettre des maths rapidement dans le texte $u_2 = 273.29 \ m.s^{-1}$. Par exemple ici un petit tips sur comment espacer rapidement les unités ! Bon en l'occurrence on aurait aussi pu utiliser \si{\meter\per\second} mais bon, qui peut le plus peut le moins.

\section{Les codes}

\begin{longlisting}
\begin{minted}{python}
import numpy as np
import math
from matplotlib import pyplot as plt

# Creation de notre structure et discretisation
def generation_structure(r_min, r_max, nnr, nntheta, oval, R, theta):
    # Creation des matrices x et y
    x = np.zeros((nntheta, nnr))
    y = np.zeros((nntheta, nnr))
    for j in range(0, nnr):
        for i in range(0, nntheta):
            x[i, j] = (R[j]+(oval/2.0))*math.cos(theta[i])
            y[i, j] = (R[j]-(oval/2.0))*math.sin(theta[i])
    return x, y

# Generation du maillage triangulaire
def maillage(x,y,nntheta,nnr,netheta,ner):
    # Creation des matrices de coordonnées et de la table de connection
    coord = np.zeros((nntheta*nnr, 2))
    table = np.zeros((2*netheta*ner, 3), dtype='int')
    # Creation des variables
    Ne = 0
    i = 0
    j = 0

    while j < nntheta-1:
        i = 0
        while i < nnr:
            if 0 < i and (Ne+1) % 2 == 0:  # Triangle avec une base en haut
                A1 = j*nnr+i
                xA1 = x[j, i]
                yA1 = y[j, i]
                A2 = (j+1)*nnr+i
                xA2 = x[j+1, i]
                yA2 = y[j+1, i]
                A3 = (j+1)*nnr+i-1
                xA3 = x[j+1, i-1]
                yA3 = y[j+1, i-1]
                if i == nnr-1:   # On arrête d'avancer, pour faire le dernier triangle supérieur
                    i = i+1
            elif 0 <= i < nnr:   # Triangle avec une base en bas
                A1 = j*nnr+i
                xA1 = x[j, i]
                yA1 = y[j, i]
                A2 = j*nnr+i+1
                xA2 = x[j, i+1]
                yA2 = y[j, i+1]
                A3 = (j+1)*nnr+i
                xA3 = x[j+1, i]
                yA3 = y[j+1, i]
                i = i+1    # Avancement le long des x

            # Remplissage de la table de connectivité avec les noeuds de l'element
            table[Ne, 0] = int(A1)
            table[Ne, 1] = int(A2)
            table[Ne, 2] = int(A3)

            # Remplissage des coordonnées des points de l'element
            coord[A1, 0] = xA1
            coord[A1, 1] = yA1
            coord[A2, 0] = xA2
            coord[A2, 1] = yA2
            coord[A3, 0] = xA3
            coord[A3, 1] = yA3

            # On passe aux nombres d'elements suivants
            Ne = Ne+1

        j = j+1
    return coord, table

# Affichage du maillage généré
def affichage_maillage(coord, table, netheta, ner):
    # On affiche l'element juste créé
    plt.figure()
    for k in range(0, 2*netheta*ner):
        # On trace un trait entre chaque noeuds d'un element
        plt.axis('equal')
        plt.plot([coord[table[k, 0], 0],
        coord[table[k, 1], 0]], [coord[table[k, 0], 1],
        coord[table[k, 1], 1]], 'r')
        plt.plot([coord[table[k, 2], 0],
        coord[table[k, 1], 0]], [coord[table[k, 2], 1],
        coord[table[k, 1], 1]], 'g')
        plt.plot([coord[table[k, 0], 0],
        coord[table[k, 2], 0]], [coord[table[k, 0], 1],
        coord[table[k, 2], 1]], 'b')
\end{minted}
\caption{Génération du maillage éléments finis sous Python}
\label{lst:PT_Mesh.py}
\end{longlisting}

\mint{python}|print("hello")| % Équivalent de \minted mais plus court

On peut meme le faire sur une ligne : \mintinline{python}|print("hello")|   % Quand t'as qu'une ligne de code

\section{Les index}

Quand ton texte contient un mot\index{mot} important\index{important}, tu peux l'indexer comme ça et il apparaîtra avec sa position dans l'index.

\chapter{Séparation dans les listes}

\section{Debug}

Ce chapitre sert juste à vérifier que la séparation entre chapitre fonctionne bien dans les listes.

\begin{figure}[H]
    \centering
    \includegraphics[width=0.25\textwidth]{example-image-golden}
    \caption{Image golden}
    \label{fig:example-image-golden}
\end{figure}

\begin{figure}[H]
    \begin{subfigure}[t]{0.475\textwidth}
        \includegraphics[width=1\textwidth]{example-grid-100x100pt}
        \caption{Image grille}
        \label{subfig:example-grid}
    \end{subfigure}%
    \hfill
    \begin{subfigure}[t]{0.475\textwidth}
        \includegraphics[width=1\textwidth]{example-grid-100x100pt}
        \caption{Image grille}
        \label{subfig:example-grid_bis}
    \end{subfigure}
    \caption{Example d'utilisation des sous-figures}
    \label{fig:test_subfigure_bis}
\end{figure}

\begin{table}[H]
\centering
\begin{tabular}{|ll|r|}
\hline
\multicolumn{2}{|c|}{\textit{\textbf{Item}}}                                      & \textit{\textbf{}}           \\ \hline
\multicolumn{1}{|l|}{\textit{\textbf{Animal}}}    & \textit{\textbf{Description}} & \textit{\textbf{Price (\$)}} \\ \hline
\multicolumn{1}{|l|}{\textit{\textbf{Gnat}}}      & \textit{\textbf{per gram}}    & \textit{\textbf{13.65}}      \\ \hline
\multicolumn{1}{|l|}{\textit{\textbf{}}}          & \textit{\textbf{each}}        & \textit{\textbf{0.01}}       \\ \hline
\multicolumn{1}{|l|}{\textit{\textbf{Gnu}}}       & \textit{\textbf{stuffed}}     & \textit{\textbf{92.50}}      \\ \hline
\multicolumn{1}{|l|}{\textit{\textbf{Emu}}}       & \textit{\textbf{stuffed}}     & \textit{\textbf{33.33}}      \\ \hline
\multicolumn{1}{|l|}{\textit{\textbf{Armadillo}}} & \textit{\textbf{frozen}}      & \textit{\textbf{8.99}}       \\ \hline
\end{tabular}
\caption{Test de tableau}
\label{tab:test_tableau}
\end{table}

\noteworthy{a+b+c=d}{Une équation très utile}

\noteworthy{\Delta=b^2-4ac}{La formule utile à tout niveau}

\begin{listing}[H]
\begin{minted}{c++}
    /*
     * Once upon a time... In a galaxy far away from ou 24 hours block
     */

    class Vampire {
      constructor(props) {
        this.location = props.location;
        this.birthDate = props.birthDate;
        this.deathDate = props.deathDate;
        this.weaknesses = props.weaknesses;
      }

      get age() {
        return this.calcAge();
      }

      calcAge() {
        return this.deathDate - this.birthDate;
      }
    }

    // ...there was a guy named Vlad 666

    const Dracula = new Vampire({
      location: 'Transylvania',
      birthDate: 1428,
      deathDate: 1476,
      weaknesses: ['Sunlight', 'Garlic']
    });
\end{minted}
\caption{Dracula sample.c}
\label{lst:dracula.c}
\end{listing}

\begin{listing}[H]
\begin{minted}{c++}
    #include <stdio.h>

    // Ceci est un test de validation du linebreak automatique des codes à l'aide du package minted. Veuillez ne pas faire attention. Ou un petit peu mais pas trop.

    int main() {

    int number1, number2, sum;
    number1 = 5;

    for(i=0,i<5,i++)
    {
        a=b+c+12.5;
        d=1e-4;
        g=124e6;
        h=sqrt(cos(x));
    }

    printf("Enter two integers: ");
    scanf("%d %d", &number1, &number2);

    // calculating sum
    sum = number1 + number2;

    printf("%d + %d = %d", number1, number2, sum);
    return 0;
\end{minted}
\caption{sample.c}
\label{lst:sample.c}
\end{listing}

Et un exemple\index{exemple} d'index pour la route !

%*-----------------------------------------------------------------------------*
%* Fin du corps du document
%*-----------------------------------------------------------------------------*

%*-----------------------------------------------------------------------------*
%* Début listes & Co
%*-----------------------------------------------------------------------------*

\newpage
\listoffigures

\newpage
\listoftables

\newpage
\listofmyequations

\newpage
\listoflistings

\newpage
\printindex

\newpage
\printbibliography[heading=bibintoc]

%*-----------------------------------------------------------------------------*
%* Fin listes & Co
%*-----------------------------------------------------------------------------*

%*-----------------------------------------------------------------------------*
%* Début des annexes
%*-----------------------------------------------------------------------------*

\begin{appendices}

\chapter*{Table des annexes}
\startcontents[chapter]
\printcontents[chapter]{l}{0}{\setcounter{tocdepth}{3}}

\chapter{Première annexe}

\cite{CitekeyBook}

\includepdf[pages=-,width=10cm]{example-image-a4}

\end{appendices}

%*-----------------------------------------------------------------------------*
%* Fin des annexes
%*-----------------------------------------------------------------------------*

\end{document}

%*=============================================================================*
%* Fin du document LaTeX
%*=============================================================================*